\documentclass[dvips,12pt]{article}

% Any percent sign marks a comment to the end of the line

% Every latex document starts with a documentclass declaration like this
% The option dvips allows for graphics, 12pt is the font size, and article
%   is the style

\usepackage[pdftex]{graphicx}
\usepackage{url}
\usepackage{cite}
\usepackage{indentfirst}

% These are additional packages for "pdflatex", graphics, and to include
% hyperlinks inside a document.

\setlength{\oddsidemargin}{0.25in}
\setlength{\textwidth}{6.5in}
\setlength{\topmargin}{0in}
\setlength{\textheight}{8.5in}

% These force using more of the margins that is the default style

\begin{document}

% Everything after this becomes content
% Replace the text between curly brackets with your own

\title{\textbf{CSE 191 Proposal}}
\author{Wesley Febrian, Jiani Huang, Soomin Lee, Yijun Zhang}
\date{\today}

% You can leave out "date" and it will be added automatically for today
% You can change the "\today" date to any text you like


\maketitle

% This command causes the title to be created in the document

\section{Research context and problem statement}

% An article style is separated into sections and subsections with --
%   markup such as this.  Use \section*{Principles} for unnumbered sections.
Learning a new paradigm in programming languages can be challenging. We are interested in analyzing students’ learning process, particularly, when learning functional programming. We have gathered data from many undergraduate students at UCSD when taking a functional programming class using OCAML, CSE 130. We want to find out where exactly students face problems or fail when doing homework, such as the types of errors that they encounter, the amount of time taken for them to fix a problem, and instances where students ‘give-up’ in doing their homework in order to help improve teaching about functional programming.


\section{Proposed solution}

In order to solve the problems mentioned above, there are several basic questions we first want to answer: 

\begin{itemize}
	
	\item How long does it take students to complete homeworks and where do the errors show up? 
	
	\begin{itemize}
		\item By answering this question, we can have a general idea on how hard each problem is for the students, and therefore knowing what type of errors should we focus on.
	\end{itemize}
	
	\item What kind of errors do the students encounter?
	\begin{itemize}
		\item If the compiler is not friendly in telling students exactly what is wrong for this kind of error, then we may do something to improve students' learning experience.
	\end{itemize}
	
	\item How long does it take them to fix the error?
	\begin{itemize}
		\item We will be able to see whether students get better over time if they spend less time to fix the error
	\end{itemize}
	
\end{itemize}

We have recorded students’ activity when doing homework in functional programming language by taking a snapshot every 30 seconds to observe any problems that a student may have--such as getting stuck on a particular line, or running the code with an error. The recorded data is compiled into JSON files. We will write Python scripts to analyze data to answer those questions.


\section{Evaluation and Implementation plan}

\paragraph{Evaluation}
The data was collected and compiled into JSON files and each log is composed of five different sections: unix timestamp, filename, file contents, offset into file contents, and event where event is one of six different kinds of objects (eval, abort, restart, timer, confused or submit). First, using Python, the data will be anonymized by converting text in the string literals to repetitions of single character to protect the privacy of the students.

\paragraph{Timeline}

\begin{itemize} 
	\item FALL 2015
	
	\begin{itemize}
		\item Get familiar with the techniques that we will be using for the following quarters, i.e. Python and OCaml.
		\item Help implement the functions needed for the code as assigned by Prof. Jhala
	\end{itemize}

\end{itemize}


\end{document}
