\documentclass[dvips,12pt]{article}

% Any percent sign marks a comment to the end of the line

% Every latex document starts with a documentclass declaration like this
% The option dvips allows for graphics, 12pt is the font size, and article
%   is the style

\usepackage[pdftex]{graphicx}
\usepackage{url}
\usepackage{cite}
\usepackage{indentfirst}

% These are additional packages for "pdflatex", graphics, and to include
% hyperlinks inside a document.

\setlength{\oddsidemargin}{0.25in}
\setlength{\textwidth}{6.5in}
\setlength{\topmargin}{0in}
\setlength{\textheight}{8.5in}

% These force using more of the margins that is the default style

\begin{document}

% Everything after this becomes content
% Replace the text between curly brackets with your own

\title{\textbf{CSE 191 Proposal}}
\author{Wesley Febrian, Jiani Huang, Soomin Lee, Yijun Zhang}
\date{\today}

% You can leave out "date" and it will be added automatically for today
% You can change the "\today" date to any text you like


\maketitle

% This command causes the title to be created in the document

\section{Research context and problem statement}

% An article style is separated into sections and subsections with 
%   markup such as this.  Use \section*{Principles} for unnumbered sections.
When the beginner of computer science try to write codes in an unfamiliar programming language, especially turns from object oriented language to functional programming language, there will be a lot of unexpected problems result from the switching between the languages. In order to observe and avoid these issues, the graduate students recorded the homework  activities by three cse 130 students, and trying to analysis how to help the students avoid the problems by add-ons. We are going to be the small group sample for testing the add-on, for both basic programming language studying purpose and for the testing aim. Also, we are going to help them in data analysis part.


Students are troubled with the different characters of FB from the OOP based language which they have already studied. Hence, some graduate students are developing an add-on to alleviate the toughness. Our mission is to evaluate the effectiveness of the add-on and study about programming language at the same time. We only have limited knowledge on the Functional Programming language, but we need to solve the high-level problems. Therefore, we are going to study the material for cse130, and get acquaintance to the developing environment. Then, we are going to do the programming assignments, to check the effectiveness of the add-on. At the same time, we help the graduate student with the data analysis job as well.

\section{Proposed solution}

Studying cse130 is a good way to start our research. We can get access to Prof. Jhala’s cse130 piazza right now, and will be able to access podcast of the class soon. We can probably catch up the course and study it in a shorter period. Also, the graduate students and professor helped us in break the analysis part into something we are able to help with, such as make the files anonymous. 

\section{Evaluation and Implementation plan}

\paragraph{Evaluation}
This project is something we can help with and benefit from. 


\paragraph{Timeline}
I think one of the most important things is we need to master the cse130 skills in a short period, so we need a well-designed timeline to go with the course. 


\end{document}
